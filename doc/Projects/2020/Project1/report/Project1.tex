%%
%% Automatically generated file from DocOnce source
%% (https://github.com/hplgit/doconce/)
%%
%%


%-------------------- begin preamble ----------------------

\documentclass[%
oneside,                 % oneside: electronic viewing, twoside: printing
final,                   % draft: marks overfull hboxes, figures with paths
10pt]{article}

\listfiles               %  print all files needed to compile this document

\usepackage{relsize,makeidx,color,setspace,amsmath,amsfonts,amssymb,
physics,mathtools,algpseudocode}

\usepackage[table]{xcolor}
\usepackage{bm,ltablex,microtype}

\usepackage[pdftex]{graphicx}

\usepackage{fancyvrb} % packages needed for verbatim environments

\usepackage[T1]{fontenc}
%\usepackage[latin1]{inputenc}
\usepackage{ucs}
\usepackage[utf8x]{inputenc}

\usepackage{lmodern}         % Latin Modern fonts derived from Computer Modern

% Hyperlinks in PDF:
\definecolor{linkcolor}{rgb}{0,0,0.4}
\usepackage{hyperref}
\hypersetup{
    breaklinks=true,
    colorlinks=true,
    linkcolor=linkcolor,
    urlcolor=linkcolor,
    citecolor=black,
    filecolor=black,
    %filecolor=blue,
    pdfmenubar=true,
    pdftoolbar=true,
    bookmarksdepth=3   % Uncomment (and tweak) for PDF bookmarks with more levels than the TOC
    }
%\hyperbaseurl{}   % hyperlinks are relative to this root

\setcounter{tocdepth}{2}  % levels in table of contents

% --- fancyhdr package for fancy headers ---
\usepackage{fancyhdr}
\fancyhf{} % sets both header and footer to nothing
\renewcommand{\headrulewidth}{0pt}
\fancyfoot[LE,RO]{\thepage}
% Ensure copyright on titlepage (article style) and chapter pages (book style)
\fancypagestyle{plain}{
  \fancyhf{}
  \fancyfoot[C]{{\footnotesize \copyright\ 1999-2020, "Computational Physics I FYS3150/FYS4150":"http://www.uio.no/studier/emner/matnat/fys/FYS3150/index-eng.html". Released under CC Attribution-NonCommercial 4.0 license}}
%  \renewcommand{\footrulewidth}{0mm}
  \renewcommand{\headrulewidth}{0mm}
}
% Ensure copyright on titlepages with \thispagestyle{empty}
\fancypagestyle{empty}{
  \fancyhf{}
  \fancyfoot[C]{{\footnotesize \copyright\ 1999-2020, "Computational Physics I FYS3150/FYS4150":"http://www.uio.no/studier/emner/matnat/fys/FYS3150/index-eng.html". Released under CC Attribution-NonCommercial 4.0 license}}
  \renewcommand{\footrulewidth}{0mm}
  \renewcommand{\headrulewidth}{0mm}
}

\pagestyle{fancy}


% prevent orhpans and widows
\clubpenalty = 10000
\widowpenalty = 10000

% --- end of standard preamble for documents ---


% insert custom LaTeX commands...

\raggedbottom
\makeindex
\usepackage[totoc]{idxlayout}   % for index in the toc
\usepackage[nottoc]{tocbibind}  % for references/bibliography in the toc

%-------------------- end preamble ----------------------

\begin{document}

% matching end for #ifdef PREAMBLE

\newcommand{\exercisesection}[1]{\subsection*{#1}}


% ------------------- main content ----------------------



% ----------------- title -------------------------

\thispagestyle{empty}

\begin{center}
{\LARGE\bf
\begin{spacing}{1.25}
Project 1, deadline  September 9
\end{spacing}
}
\end{center}

% ----------------- author(s) -------------------------

\begin{center}
{\bf \href{{http://www.uio.no/studier/emner/matnat/fys/FYS3150/index-eng.html}}{Computational Physics I FYS3150/FYS4150}}
\end{center}

    \begin{center}
% List of all institutions:
\centerline{{\small Department of Physics, University of Oslo, Norway}}
\end{center}
    
% ----------------- end author(s) -------------------------

% --- begin date ---
\begin{center}
Aug 25, 2020
\end{center}
% --- end date ---

\vspace{1cm}


\subsection*{Introduction}
The aim of this project is to get familiar with various vector and matrix operations,
from dynamic memory allocation to the usage of programs in the library
package of the course. 
For Fortran users memory handling and most matrix and vector operations
are included in the ANSI standard of Fortran 90/95. Array handling in Python is also rather trivial. For C++ user however,
there are several possible options. Two are listed here.

\begin{itemize}
  \item For this exercise we recommend that you make your own functions for dynamic memory allocation of a  vector and a matrix. You don't need to write a class for this operations.  Use then the  library package lib.cpp with its header file  lib.hpp for obtaining LU-decomposed matrices, solve linear equations etc.

  \item A very good and often recommended library for C++ handling of arrays is the library Armadillo, to be found at \url{arma.sourceforge.net}.  We will discuss the usage of this library during the lab sessions and lectures. Armadillo has also an interface to Lapack functions for solving systems of linear equations. Alternatively you can use the \href{{http://eigen.tuxfamily.org/index.php?title=Main_Page}}{Eigen} library, which has much of the same functionality as Armadillo.
\end{itemize}

\noindent
Your program, whether it is written in C++, Python, Fortran or other languages, should include
dynamic memory handling of matrices and vectors. 

The material needed for this project is covered by chapter 6 of the lecture notes, in particular section 6.4 and subsequent sections.



Many important differential equations in  Science can be written as 
linear second-order differential equations

\begin{equation*}
\frac{d^2y}{dx^2}+k^2(x)y = f(x),
\end{equation*}
where $f$ is normally called the inhomogeneous term and $k^2$ is a real function.

A classical equation from electromagnetism is Poisson's equation.
The electrostatic potential $\Phi$ is generated by a localized charge
distribution $\rho (\mathbf{r})$.   In three dimensions 
it reads

\begin{equation*}
\nabla^2 \Phi = -4\pi \rho (\mathbf{r}).
\end{equation*}
With a spherically symmetric $\Phi$ and $\rho (\mathbf{r})$  the equations
simplifies to a one-dimensional equation in $r$, namely

\begin{equation*}
\frac{1}{r^2}\frac{d}{dr}\left(r^2\frac{d\Phi}{dr}\right) = -4\pi \rho(r),
\end{equation*}
which can be rewritten via a substitution $\Phi(r)= \phi(r)/r$ as

\begin{equation*}
\frac{d^2\phi}{dr^2}= -4\pi r\rho(r).
\end{equation*}
The inhomogeneous term $f$ or source term is given by the charge distribution
$\rho$  multiplied by $r$ and the constant $-4\pi$.

We will rewrite this equation by letting $\phi\rightarrow u$ and 
$r\rightarrow x$. 
The general one-dimensional Poisson equation reads then

\begin{equation*}
-u''(x) = f(x).
\end{equation*}


\paragraph{Project 1 a):}
We have
\begin{align*}
-\frac{v_{i+1} + v_{i-1} - 2v_i}{h^2} &= f_i \qfor i=1,...,n,
\\ -v_{i+1} - v_{i-1} + 2v_i &= h^2f_i
\shortintertext{or, on vector form,}
\bmqty{-1&2&-1}\bmqty{v_{i-1}\\v_i\\v_{i+1}} &= h^2f_i
\shortintertext{Expanding this expression to all $i$ gives}
\bmqty{	2 & -1 &  &  &  &  &  \\ 
		-1 & 2 & -1 &  &  &  & \\
		 & \ddots & \ddots & \ddots &  & 0 &  \\
		 &  & -1 & 2 & -1 &  & \\
		 & 0 & & \ddots & \ddots & \ddots & \\
		 & & & & -1 & 2 & -1 \\
		 & & & & & -1 & 2
		}
\bmqty{v_1 \\ \vdots \\ v_{i-1} \\ v_i \\ v_{i+1} \\ \vdots \\ v_{n}}
&= h^2\bmqty{f_1 \\ \vdots \\ f_{i-1} \\ f_i \\ f_{i+1} \\ \vdots \\ f_{n}}
\end{align*}
Note that the terms 0 and $n+1$ are not included because of the boundary conditions $v_0 = v_{n+1} = 0$.

\paragraph{Project 1 b):}
The first goal is to create an algorithm which can solve equations for any tridiagonal matrix. 
\[
\vb{A} = \bmqty{b_1 & c_1 & & & &\\
				a_1 & b_2 & c_2 & & 0 & \\
				& a_2 & b_3 & c_3 & & \\
				& & \ddots & \ddots & \ddots & \\
				& 0 & & a_{n-1} & b_{n-1} & c_{n-1} \\
				& & & & a_{n-1} & b_n}.
\]
We use forward substitution to eliminate the $a$-diagonal, then backward substitution to eliminate the $c$-diagonal. The algorithm to implement is then \cite{mat}
\\ 
\begin{algorithmic}
\State Forward substitution: 
\For{$i$ from 1 to $n-1$}
	\State $\tilde{b}_i = b_i - \frac{c_{i-1} a_{i-1}}{\tilde{b}_{i-1}}$
	\State $\tilde{g}_i = g_i - \frac{\tilde{g}_{i-1}a_{i-1}}{\tilde{b}_{i-1}}$
\EndFor
\State Backward substitution:
\For{$i$ from $n$ to 1}
	\State $u_i = \frac{\tilde{g}_i - c_iu_{i+1}}{\tilde{b}_i}$
\EndFor
\end{algorithmic}

When programming this, all indices start at 0, so 

\paragraph{Project 1 c):}
Next, we look at the special case where the matrix has identical matrix
elements along the diagonal and identical (but different) values for the nondiagonal
elements. In other words
\[
a_i = a\qcomma b_i = b\qcomma c_i = c \qq{for all $i$.}
\]
In this case, $\tilde{b}_i$  

\paragraph{Project 1 d):}
Compute the relative error  in the data set $i=1,\dots, n$,by setting up

\[
   \epsilon_i=log_{10}\left(\left|\frac{v_i-u_i}
                 {u_i}\right|\right),
\]
as function of $log_{10}(h)$ for the function values $u_i$ and $v_i$.
For each step length extract the max value of the relative error.  
Try to increase $n$ to $n=10^7$.  Make a table of the results and 
comment your results. You can use either the algorithm from b) or c). 

\paragraph{Project 1 e):}
Compare your results with those from the LU decomposition codes for the matrix of sizes $10\times 10$, $100\times 100$ and
$1000\times 1000$. Here you should use the library functions provided  on the webpage of the course. Alternatively, if you use armadillo as a library, you can use the similar function for LU decomposition.  The armadillo function for the LU decomposition is called $LU$ while the function for solving linear sets of equations is called $solve$.
Use for example the unix function \emph{time} when you run your codes 
and compare the time usage between LU decomposition and  your
tridiagonal solver.   Alternatively, you can use the functions in C++, Fortran or Python that measure the time used. 

Make a table of the results and comment the differences
in execution time
How many floating point operations does the LU decomposition use to solve the set of linear equations?
Can you run the standard LU decomposition
for a matrix of the size $10^5\times 10^5$?
Comment your results.


To compute the elapsed time in c++ you can use the following statements
\begin{verbatim}
...
#include "time.h"   //  you have to include the time.h header
int main()
{
    // declarations of variables 
    ...
    clock_t start, finish;  //  declare start and final time
    start = clock();
    // your code is here, do something and then get final time
    finish = clock();
    ( (finish - start)/CLOCKS_PER_SEC );
...
\end{verbatim}
Similarly, in Fortran, this simple example shows how to compute the elapsed time.
\begin{verbatim}
PROGRAM time
 REAL :: etime          ! Declare the type of etime()
 REAL :: elapsed(2)     ! For receiving user and system time
 REAL :: total          ! For receiving total time
 INTEGER :: i, j

 WRITE(*,*) 'Start'

 DO i = 1, 5000000  
      j = j + 1
 ENDDO

 total = ETIME(elapsed)
 WRITE(*,*) 'End: total=', total, ' user=', elapsed(1), &
              ' system=', elapsed(2)

END PROGRAM time
\end{verbatim}

Your results may depend on the granularity of the clock.




\subsection*{Introduction to numerical projects}

Here follows a brief recipe and recommendation on how to write a report for each
project.

\begin{itemize}
  \item Give a short description of the nature of the problem and the eventual  numerical methods you have used.

  \item Describe the algorithm you have used and/or developed. Here you may find it convenient to use pseudocoding. In many cases you can describe the algorithm in the program itself.

  \item Include the source code of your program. Comment your program properly.

  \item If possible, try to find analytic solutions, or known limits in order to test your program when developing the code.

  \item Include your results either in figure form or in a table. Remember to        label your results. All tables and figures should have relevant captions        and labels on the axes.

  \item Try to evaluate the reliabilty and numerical stability/precision of your results. If possible, include a qualitative and/or quantitative discussion of the numerical stability, eventual loss of precision etc.

  \item Try to give an interpretation of you results in your answers to  the problems.

  \item Critique: if possible include your comments and reflections about the  exercise, whether you felt you learnt something, ideas for improvements and  other thoughts you've made when solving the exercise. We wish to keep this course at the interactive level and your comments can help us improve it.

  \item Try to establish a practice where you log your work at the  computerlab. You may find such a logbook very handy at later stages in your work, especially when you don't properly remember  what a previous test version  of your program did. Here you could also record  the time spent on solving the exercise, various algorithms you may have tested or other topics which you feel worthy of mentioning.
\end{itemize}

\noindent
\subsection*{Format for electronic delivery of report and programs}

The preferred format for the report is a PDF file. You can also use DOC or postscript formats or as an ipython notebook file.  As programming language we prefer that you choose between C/C++, Fortran2008 or Python. The following prescription should be followed when preparing the report:

\begin{itemize}
  \item Use \textbf{Canvas} to hand in your projects, log in  at  \href{{https://www.uio.no/english/services/it/education/canvas/}}{\nolinkurl{https://www.uio.no/english/services/it/education/canvas/}} with your normal UiO username and password.

  \item Upload \textbf{only} the report file!  For the source code file(s) you have developed please provide us with your link to your github domain.  The report file should include all of your discussions and a list of the codes you have developed.  Do not include library files which are available at the course homepage, unless you have made specific changes to them. Alternatively, you can just upload the address to your GitHub or GitLab repository.

  \item In your git repository, please include a folder which contains selected results. These can be in the form of output from your code for a selected set of runs and input parameters.

  \item In this and all later projects, you should include tests (for example unit tests) of your code(s).

  \item Comments  from us on your projects, approval or not, corrections to be made  etc can be found under your \textbf{Canvas} domain and are only visible to you and the teachers of the course.
\end{itemize}

\noindent
Finally, 
we encourage you to work two and two together. Optimal working groups consist of 
2-3 students. You can then hand in a common report. 












% ------------------- end of main content ---------------

\end{document}

