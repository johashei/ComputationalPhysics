%%
%% Automatically generated file from DocOnce source
%% (https://github.com/hplgit/doconce/)
%%
%%
% #ifdef PTEX2TEX_EXPLANATION
%%
%% The file follows the ptex2tex extended LaTeX format, see
%% ptex2tex: http://code.google.com/p/ptex2tex/
%%
%% Run
%%      ptex2tex myfile
%% or
%%      doconce ptex2tex myfile
%%
%% to turn myfile.p.tex into an ordinary LaTeX file myfile.tex.
%% (The ptex2tex program: http://code.google.com/p/ptex2tex)
%% Many preprocess options can be added to ptex2tex or doconce ptex2tex
%%
%%      ptex2tex -DMINTED myfile
%%      doconce ptex2tex myfile envir=minted
%%
%% ptex2tex will typeset code environments according to a global or local
%% .ptex2tex.cfg configure file. doconce ptex2tex will typeset code
%% according to options on the command line (just type doconce ptex2tex to
%% see examples). If doconce ptex2tex has envir=minted, it enables the
%% minted style without needing -DMINTED.
% #endif

% #define PREAMBLE

% #ifdef PREAMBLE
%-------------------- begin preamble ----------------------

\documentclass[%
oneside,                 % oneside: electronic viewing, twoside: printing
final,                   % draft: marks overfull hboxes, figures with paths
10pt]{article}

\listfiles               %  print all files needed to compile this document

\usepackage{relsize,makeidx,color,setspace,amsmath,amsfonts,amssymb}
\usepackage[table]{xcolor}
\usepackage{bm,ltablex,microtype}

\usepackage[pdftex]{graphicx}

\usepackage[T1]{fontenc}
%\usepackage[latin1]{inputenc}
\usepackage{ucs}
\usepackage[utf8x]{inputenc}

\usepackage{lmodern}         % Latin Modern fonts derived from Computer Modern

% Hyperlinks in PDF:
\definecolor{linkcolor}{rgb}{0,0,0.4}
\usepackage{hyperref}
\hypersetup{
    breaklinks=true,
    colorlinks=true,
    linkcolor=linkcolor,
    urlcolor=linkcolor,
    citecolor=black,
    filecolor=black,
    %filecolor=blue,
    pdfmenubar=true,
    pdftoolbar=true,
    bookmarksdepth=3   % Uncomment (and tweak) for PDF bookmarks with more levels than the TOC
    }
%\hyperbaseurl{}   % hyperlinks are relative to this root

\setcounter{tocdepth}{2}  % levels in table of contents

% prevent orhpans and widows
\clubpenalty = 10000
\widowpenalty = 10000

% --- end of standard preamble for documents ---


% insert custom LaTeX commands...

\raggedbottom
\makeindex
\usepackage[totoc]{idxlayout}   % for index in the toc
\usepackage[nottoc]{tocbibind}  % for references/bibliography in the toc

%-------------------- end preamble ----------------------

\begin{document}

% matching end for #ifdef PREAMBLE
% #endif

\newcommand{\exercisesection}[1]{\subsection*{#1}}


% ------------------- main content ----------------------



% ----------------- title -------------------------

\thispagestyle{empty}

\begin{center}
{\LARGE\bf
\begin{spacing}{1.25}
Project 5, deadline  December 16, 2020
\end{spacing}
}
\end{center}

% ----------------- author(s) -------------------------

\begin{center}
{\bf Black-Scholes equation${}^{}$} \\ [0mm]
\end{center}

\begin{center}
% List of all institutions:
\end{center}
    
% ----------------- end author(s) -------------------------

% --- begin date ---
\begin{center}
Fall semester 2020
\end{center}
% --- end date ---

\vspace{1cm}


\subsection{Solving the Black-Scholes Equation Numerically}

WARNING: this is a somewhat experimental problem given for the 
first time the Autumn of 2020. Only a very minimal theoretical 
background is given in this text. Feel free to write to Sebastian for more information.

\paragraph{The briefest introduction to options.}
A financial derivative is a contract that derives its value from 
the performance of an underlying asset. Perhaps the most common 
financial derivative is the option. Such contracts are 
incredibly old, the first mention of options contracts in 
history is by greek philosopher Thales from the sixth century.

An option is a right, but not an obligation, to buy or sell 
and underlying asset\footnote{We focus on stocks, but it can be anything.} at a predetermined price $E$ at or before 
an expiration time $T$. Having such an option is valuable, but
determining the fair price of an option is a difficult problem.



Options to buy (sell) are commonly referred to as 
call (put) options. An option that only allows one to 
excercise this right at the maturity date is called a 
European option, while an option that can be exercise at any 
date prior to the maturity date is called an American option.
There are many other varieties.

In 1997, the (pseudo-)Nobel prize in economics was awarded 
to Robert C. Merton and Myron S. Scholes "for a new method
to determine the value of derivatives", the Black-Scholes-Merton
model.

\paragraph{The Black-Scholes equation.}
The Black-Scholes equation is a partial differential equation, which 
describes the price of an option over time. The key insight behind the 
equation is that one can perfectly hedge the option by 
buying and selling the underlying asses and the "bank account 
asset" (cash) in just the right way to eliminate risk.

\begin{equation}
    \frac{\partial V}{\partial t}
    + \frac{1}{2}S^2\sigma^2\frac{\partial^2 V}{\partial S^2}
    + (r - D)S\frac{\partial V}{\partial S} - r V = 0
\end{equation}

Here $V(S, T)$ is the value of the options, $S$ is the price of the 
underlying asset, $\sigma$ is the volatility of the underlying asset,
$r$ is the "risk-free" interest rate, and $D$ is the yield
(dividend paying rate) of the underlying stock.

The volatility $\sigma$ stems from an underlying assumption that 
the stock moves like a geometric Brownian motion,
\begin{equation}
    \frac{dS}{S} = \mu dt + \sigma dW.
\end{equation}

Explicit solutions for the Black-Scholes equation,
called The Black-Scholes formulae, are known only for 
European call and put options. For other derivatives, such 
a formula doest not have to exist. However, a numerical solution is 
always possible.

\paragraph{5a: Transformation to Heat Equation/Diffusion equation.}
Instead of an initial value problem, we have a terminal value
problem at time $T$, i.e.~the expiration date or the 
maturity date. We change to an initial value problem 
by substituting $\tau = T - t$. This new variable can 
be interpreted as time remaining to expiration.

The transformed spatial variable is $x = \ln(S/E)$, where 
$E$ is the exercise price of the option. Now, values of 
$x$ close to zero correspond to stock prices that are close to 
the exercise price of the option. Negative values of $x$ 
correspond to stock prices lower than the exercise price and 
positive values of $x$ correspond to prices higher than the 
exercise price.

Just substituting for the variables above leads to a parabolic 
equation, with constant coefficients. Show that by making a final
substitution;
\begin{equation}
    u(x, \tau) = e^{\alpha x + \beta \tau} V(S, t)
\end{equation}
we get the heat (diffusion) equation
\begin{equation}
    \frac{\partial u}{\partial \tau}
    = \frac{\sigma^2}{2}\frac{\partial^2 u}{\partial x^2}
\end{equation}

What are the correct parameters for $\alpha$ and $\beta$?

% SUPPOSED TO BE COMMENTED. THIS IS THE SOLUTION.

% $$
% \alpha = \frac{r - D}{\sigma^2} - \frac{1}{2},
% $$

% $$
% \beta = \frac{r - D}{2} + \frac{\sigma^2}{8}
% + \frac{(r - D)^2}{2\sigma^2}.
% $$

\paragraph{5b: Create solver(s) for the 1D diffusion equation.}
You can implement a solver for the diffusion equation 
inspiration from the project on the diffusion equation.
Another very good resource is Langtangen and Linge's 
book "Finite Difference Computing with PDEs".

It is highly recommended to start with an explicit 
scheme, and then move to a more sophisticated 
implicit scheme. You should make up your 
own mind regarding what scheme is robust enough.
Gauss-Seidel (successive over-relaxation) and
Crank-Nicholson are some suggestions.

After creating a general diffusion equation solver, you
can adapt the program to solve the Black-Scholes equation.
Plot $V$ vs $S$ at different values for $t$ ($\tau$).

\paragraph{Special considerations}
The variable $x$ is unbounded, i,e. $x\in[-\infty, \infty]$. Numerically,
we have to pick a bounded interval $[-L, L]$, where $L$ is a sufficiently 
large number. This interval remain unchanged when considering an option with a
different strike price.

You need to impose special boundary conditions for $x=-L$ and $x=L$, corresponding 
to stock prices that are close to zero and very high, approaching infinity.

You need to pick som values for $E$, $r$, $D$ and $\sigma$ some starting
choices can $50$, $0.04$, $0.12$ and $0.4$, respectively. Discuss what 
values would be reasonable - in the present market situation the values 
suggested are absolutely not!

In order to make reasonable plots, you need to transform the solution of 
the diffusion equation into the solution for the Black-Scholes 
equation. 

You should compare to the analytic solution, i.e.~the Black-Scholes Formula:

\begin{equation}
    C(S_t, t) = N(d_1) S_t - N(d_2) PV(E),
\end{equation}
where the present value of the exercise price is
given by
\begin{equation}
    PV(E) = Ee^{-rt},
\end{equation}
furthermore we have parameters $d_1$,
\begin{equation}    
    d_1 = \frac{1}{\sigma \sqrt{T - t}}
    \left[
        \ln \left(\frac{S_t}{E} \right) 
        + \left(r + \frac{\sigma^2}{2} \right) (T - t),
    \right] 
\end{equation}
and $d_2$,
\begin{equation}
    d_2 = d_1 - \sigma \sqrt{t},
\end{equation}
while $N$ is the cumulative normal distribution function.


\paragraph{5c: Compute values for first-order Greeks.}
The "Greeks" measure sensitivity of the value of the derivative of 
a portfolio to changes in parameter value while holding other 
parameters fixed. There are different ways of hedging a portfolio
agains risk by eliminating the movement in one or several of
these parameters.

Delta: $\Delta = \frac{\partial V}{\partial S}$

Gamma: $\gamma = \frac{\partial^2 V}{\partial V}$

Vega\footnote{Notice that this is not really a greek letter.}: $\nu = \frac{\partial V}{\partial \sigma}$



Theta: $\Theta = -\frac{\partial V}{\partial \tau}$

Rho: $\rho = \frac{\partial V}{\partial r}$

Make a representation of how these parameters move over time.
Why are the Greeks interesting? Is your computed results reasonable?


\paragraph{5d: Find data and compute implied volatility.}
There are several market-traded standardised options. On the 
Oslo Stock Exchange (OSE) there are standardised options for several 
of the larger listed companies such as Equinor, Norsk Hydro,
Telenor, Mowi, Orkla etc. See for instance
\href{{https://www.oslobors.no/markedsaktivitet/#/list/derivatives/quotelist/false}}{\nolinkurl{https://www.oslobors.no/markedsaktivitet/\#/list/derivatives/quotelist/false}}

Since there the market already prices the options, one can use these options 
prices to derive the value of implicit variables. A common practice 
is to use the market-given prices of options to derive the 
implied volatility of the option. 

To do this; pick a large company with a high turnover of stock,
and whose stock has seen several actual option trades on OSE lately.
Find a reasonable estimate for the risk-free interest rate $r$. This would 
typically be given by short-term government bonds of the same currency as 
the stock. In NOK, you can find "Statskassaveksler" - these are bonds issued 
by the Norwegian Government with a shorter maturity than 12 months.
Find the effective, real, annualised interest of these and use that as an 
estimate for $r$ (these days it would be very close to zero). Norges 
Bank publishes data on this. You can also find quotes on OSE.
The dividend yield $D$ of the stock can be computed by taking the sum 
of all dividends paid py the company per stock for the year and dividing
by the stock 
price. Then you should be able to tune the volatility in your solver to 
fit the call prices given by the market. You can do this by trial and 
error, but also use a root-finder like Newton's method or Brent's 
method.


\subsection{Introduction to numerical projects}

Here follows a brief recipe and recommendation on how to write a report for each
project.

\begin{itemize}
  \item Give a short description of the nature of the problem and the eventual  numerical methods you have used.

  \item Describe the algorithm you have used and/or developed. Here you may find it convenient to use pseudocoding. In many cases you can describe the algorithm in the program itself.

  \item Include the source code of your program. Comment your program properly.

  \item If possible, try to find analytic solutions, or known limits in order to test your program when developing the code.

  \item Include your results either in figure form or in a table. Remember to        label your results. All tables and figures should have relevant captions        and labels on the axes.

  \item Try to evaluate the reliability and numerical stability/precision of your results. If possible, include a qualitative and/or quantitative discussion of the numerical stability, eventual loss of precision etc.

  \item Try to give an interpretation of you results in your answers to  the problems.

  \item Critique: if possible include your comments and reflections about the  exercise, whether you felt you learnt something, ideas for improvements and  other thoughts you've made when solving the exercise. We wish to keep this course at the interactive level and your comments can help us improve it.

  \item Try to establish a practice where you log your work at the  computerlab. You may find such a logbook very handy at later stages in your work, especially when you don't properly remember  what a previous test version  of your program did. Here you could also record  the time spent on solving the exercise, various algorithms you may have tested or other topics which you feel worthy of mentioning.
\end{itemize}

\noindent
\subsection{Format for electronic delivery of report and programs}

The preferred format for the report is a PDF file. You can also use DOC or postscript formats or as an ipython notebook file.  As programming language we prefer that you choose between C/C++, Fortran2008 or Python. The following prescription should be followed when preparing the report:

\begin{itemize}
  \item Use \textbf{Canvas} to hand in your projects, log in  at  \href{{https://www.uio.no/english/services/it/education/canvas/}}{\nolinkurl{https://www.uio.no/english/services/it/education/canvas/}} with your normal UiO username and password.

  \item Upload \textbf{only} the report file!  For the source code file(s) you have developed please provide us with your link to your github domain.  The report file should include all of your discussions and a list of the codes you have developed.  Do not include library files which are available at the course homepage, unless you have made specific changes to them. Alternatively, you can just upload the address to your GitHub or GitLab repository.

  \item In your git repository, please include a folder which contains selected results. These can be in the form of output from your code for a selected set of runs and input parameters.

  \item In this and all later projects, you should include tests (for example unit tests) of your code(s).

  \item Comments  from us on your projects, approval or not, corrections to be made  etc can be found under your \textbf{Canvas} domain and are only visible to you and the teachers of the course.
\end{itemize}

\noindent
Finally, 
we encourage you to work two and two together. Optimal working groups consist of 
2-3 students. You can then hand in a common report. 

















% ------------------- end of main content ---------------

% #ifdef PREAMBLE
\end{document}
% #endif

