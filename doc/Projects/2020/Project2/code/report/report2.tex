% Dokument:
\documentclass[a4paper,10pt,twocolumn]{article}
\usepackage[utf8]{inputenc}
%\usepackage[norsk]{babel}
\usepackage{hyperref}

% Formatering:
\usepackage{geometry,amsthm,blindtext}
\setlength{\parindent}{3mm}
\setlength{\parskip}{1.8mm}
% Symboler
\usepackage[geometry]{ifsym}
% Matematikk:
\usepackage{amsmath,mathtools,amsfonts,xparse,physics,esint,mathrsfs,nicefrac,siunitx}
% Kode:
\usepackage{verbatim,listings}
% Tekst:
\usepackage{textcomp,varioref,enumerate,bbold}
% Figurer:
\usepackage{graphicx,subcaption,float,dcolumn,multirow}
\usepackage[section]{placeins}
% Farge:
\usepackage{xcolor}

% Dcolumn-kolonner
\newcolumntype{,}{D{,}{{,}}{2}}
\newcolumntype{=}{D{=}{=}{-1}}
\newcolumntype{.}{D{.}{{.}}{2}}
\newcolumntype{C}{>{$}c<{$}}

% Listings-Instillinger
	\lstset{
	% Språk
	language=c++,
	% Farge
	backgroundcolor=	\color[rgb]{0.9,0.9,0.9},
	keywordstyle=		\color[rgb]{0,0,1},
        	commentstyle=		\color[rgb]{0.133,0.545,0.133},
        	stringstyle=		\color[rgb]{0.627,0.126,0.941},
	numberstyle=\tiny	\color[rgb]{0.5,0.5,0.5},
        	rulecolor=			\color{black},
	% Tekst:
	basicstyle=\scriptsize,
	showspaces=false,
	showstringspaces=false,
	showtabs=false,
	% Spesialtegn
	extendedchars=true,
        	literate=	{æ}{{\ae}}1 	{ø}{{\o}}1 		{å}{{\r a}}1 
		  	{≤}{{$\leq$}}1 	{≥}{{$\geq}}1	{–}{{$-$}}1		{~}{{\tiny{$\sim$}}}1,
        	% Format
	frame=single,
	aboveskip={1.5\baselineskip},
	breaklines=true,
	columns=fixed,
	numbers=left,
	numbersep=5pt,
	stepnumber=1,
	tabsize=4, 
	% ... 
        	upquote=true,
        	prebreak = \raisebox{0ex}[0ex][0ex]{\ensuremath{\hookleftarrow}},
        	identifierstyle=\ttfamily
        	}
	
% Nummerering:
%\renewcommand{\thesection}{\arabic{section}}		% Nummer i "section"
%\renewcommand{\thessubection}\arabic{subsection}}	% Nummer i "subsection"
%\renewcommand{\thesection}{\alph{section}}			% Bokstaver i ''section''
%\renewcommand{\thesubsection}{\alph{subsection}}	% Bokstaver i ''subsection''

% Egendefinerte kommandoer:
	% Snarveier
	\newcommand*{\eqset}[1]{\begin{dcases}#1\end{dcases}}
	\newcommand*{\Qfig}[3]{\begin{figure}[ht]	\centering \includegraphics[#2]{#1} \caption{#3} \end{figure} }
	% Tegn
	\newcommand{\point}{\varointclockwise}	% integral med klokken
	\newcommand{\noint}{\ointctrclockwise}	% integral mot klokken
	\newcommand{\ee}[1]{\!\times\!\!10^{#1}} 	% 10 i #1-te
	\newcommand{\un}[1]{\,\si{#1}}		% enheter
	\newcommand{\e}[1]{\mathrm{e}^{#1}}	% exp()
	\newcommand{\ci}{\dot{\iota\!\:}}		% kompleks i
	% misc
	\newcommand{\note}[1]{{\color{red}\quad[$\backslash!/$: #1]}}	% notater for dokument under arbeid
	\newlength{\txtsz}\makeatletter\setlength{\txtsz}{\f@size pt}\makeatother
	\newcommand{\insubsec}[1]{\par{\bf #1}}

% Åpning:
\title{Eigenvalue problems, comparison of Jacobi's and Lanczos' algorithms applied to Schroedinger’s equation for an electron in a threedimensional
harmonic oscillator well}
\author{Johannes Sørby Heines}


% START
\begin{document}

\twocolumn[
\begin{@twocolumnfalse}
\maketitle
\begin{abstract}

\end{abstract}
\end{@twocolumnfalse}
]

\section*{Introduction}

In physics, we often encounter systems whose behaviour depends on their current state. In many cases these can be described or approximated by equations of the form  
\[
\dv[2]{u(x)}{x} = \lambda u(x).
\]
When such equations are discretized they give rise to an eigenvalue problem
\[
\vb{A}u = \lambda u.
\]
There are many different algorithms for solving eigenvalue problems numerically. We explored two methods for cases in which $A$ is a real symmetrical matrix: Jacobi's and Lanczos' algorithms.  

We outline the mathematical workings of each algorithm, describe their numerical implementation in C++ and apply them to the simple case of a single electron in a harmonic oscillator potential. 

%
%
%
\section*{Theory}

Numerical solutions to eigenvalue problems make use of similarity transformations:
\[
\vb{B} = \vb{S}^T\vb{A}\vb{S}, \qq{where} \vb{S}^T\vb{S} = \mathbb{1}.
\]
These are useful because they preserve the eigenvalues \cite{lecture} and the orthogonality of the eigenvectors.
 \begin{proof}
Consider an orthogonal basis of vectors 
\[
\vb{v}_i = \bmqty{v_{i1}\\\vdots\\v_{in}}\qcomma \vb{v}_j^T\vb{v}_i = \delta_{ij}.
\]
Let \(\vb{w}_i = \vb{S}^T\vb{v}_i\vb{S}\). Then
\begin{align*}
\vb{w}_j^T\vb{w}_i &= (\vb{S}^T\vb{v}_j\vb{S})^T(\vb{S}^T\vb{v}_i\vb{S}) 
\\& = \vb{S}^T\vb{v}_j^T\vb{S}\vb{S}^T\vb{v}_i\vb{S}
\\& = \vb{S}^T\vb{v}_j^T\vb{v}_i\vb{S}
\\& = \vb{S}^T\delta_{ij}\vb{S} = \delta_{ij}.
\end{align*}

\end{proof}

\subsection*{Jacobi's algorithm}
Direct methods determine the eigenvalues of a matrix $\vb{A}$ by performing a series of similarity transformations 
\[
\vb{S}_N^T \cdots \vb{S}_1^T \vb{A} \vb{S}_1\cdots \vb{S}_N = \vb{D}, 
\]
such that $\vb{D}$ is a diagonal matrix. Because similarity transformations preserve the eigenvalues, the diagonal elements of $\vb{D}$ are the eigenvalues of $\vb{A}$. 
There is no uniquely defined series of similarity transformations which lead to the matrix $\vb{B}$. 
Jacobi's algorithm is one way of determining and performing these similarity transformations. 
Each iteration finds the maximum off-diagonal element of $\vb{A}$, and performs a rotation along an axis to set that element to zero. This systematically reduces the Frobenius norm of the off-diagonal elements
\[
\mathrm{off}(A) = \sqrt{\sum_{i=1}^n\sum_{j=1,j\neq i}^n a_{ij}^2}.
\]
When this value close enough to zero, the algorithm ends giving approximate eigenvalues of $\vb{A}$.

\subsection*{Lanczos' algorithm}

\subsection*{Single electron in a harmonic oscillator potential}
An electron in a harmonic oscillator potential can occupy energy levels given by the time independent radial Schrödinger equation \cite{labtext}
\begin{align}\label{eq:schrodinger}
-\frac{\hbar^2}{2m}&\qty(\frac{1}{r^2}\dv{r}r^2\dv{r}-\frac{l(l+1)}{r^2})R(r) 
\\ &\qquad\qquad\qquad+ V(r)R(r) = ER(r). \nonumber
\end{align} 
Here $R(r)$ is the radial wave function, 
$V(r)$ is the harmonic oscillator potential,
the quantum number $l$ is the electron's orbital angular momentum,
This is an eigenvalue problem where the eigenvalues $E$ represent the energy levels.
\note{scaling in methods}

%
%
%
\section*{Methods}
\subsection*{Jacobi's algorithm}
The Jacobi algorithm is implemented in three steps. First the program loops through the off-diagonal elements to find the maximum absolute value $\abs{a_{kl}}$. Since the matrix is symmetric it only checks the lower elements. 
Second, it calculates the rotation angle. \note{add calculation to theory} Here, equation \ref{eq:t} can lead to loss of numerical precision when $\tau^2 >> 1$, so it's implemented as
\begin{align*}
t &= \frac{(-\tau\pm\sqrt{1+\tau^2})(\pm\tau+\sqrt{1+\tau^2})}{\pm\tau+\sqrt{1+\tau^2}}
\\&= \frac{\pm1}{\pm\tau + \sqrt{1+\tau^2}}.
\end{align*}
Third, the program performs the rotation. It temporarily stores the current values of $\vb{A}$ when needed, and overwrites the matrix with the new values.   

\subsection*{Lanczo's algorithm}


\subsection*{Scaling of the Schrödinger equation}
In order to apply the algorithms above to equation \ref{eq:schrodinger}, we introduce new variables to obtain a dimensionless equation. We only look at the case where $l=0$. 

By making the substitution $R(r) = u(r)/r$, we get
\[
-\frac{\hbar^2}{2m}\dv[2]{r}u(r) + V(r)u(r) = Eu(r)
\]
We then define $\rho = t/\alpha$ where $\alpha$ is a constant with dimension length, giving $V(\rho) = k\alpha^2\rho^2/2$, and thus
\[
-\frac{\hbar^2}{2m\alpha^2}\dv[2]{\rho}u(\rho) + \frac{k}{2}\alpha^2\rho^2u(\rho) = Eu(\rho).
\] 
Now we can choose $\alpha$ so that $mk\alpha^4/\hbar^2 = 1$ and define $\lambda = 2m\alpha^2E/\hbar^2$, and rewrite the Schrödinger equation as
\[
-\dv[2]{\rho}u(\rho) + \rho^2u(\rho) = \lambda u(\rho).
\]
This equation can then be discretized, giving the eigenvalue equation $\vb{A}u = \lambda u$, where $\vb{A}$ is a tridiagonal matrix
\[
\vb{A} = 
\bmqty{	d_1 & e_1 & &  \\
		e_1	& d_2 & \ddots & \\
		 & \ddots & \ddots & e_{N-2}\\
		 & & e_{N-2} & d_{N-1}},	
\] 
with $e_i = -1/h^2$ and $d_i = 2/h^2+\rho_i^2$, where $h = (\rho_N-\rho_0)/N$ is the step length \cite{labtext}.



%
%
%
\section*{Results}
With $\rho_N = 30$ and $N=

%
%
%
\section*{Discussion}

%
%
%
\section*{Conclusion}



\onecolumn
\begin{thebibliography}{9}


\end{thebibliography}



% Husk å avslutte alle environments
% SLUTT
\end{document}